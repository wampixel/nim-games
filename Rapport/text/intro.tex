Durant la formation de licence informatique, nous avons obtenu de fortes base théoriques et pratiques et nous avons la possibilité de pouvoir mettre a profit nos connaissances durant un stage de deux mois. Le but de ce dernier est d'avoir une première approche du milieu professionnel dans le monde de l'informatique et de pouvoir évaluer nos connaissances acquises.

c'est dans ce but que j'ai obtenu un stage au sein du département informatique de l’université de Rouen.

Ce stage fut pour moi une bonne occasion de pouvoir découvrir le milieu de la recherche informatique et de pouvoir valider mes choix d'orientation. En effet, ce stage m'a permis de pouvoir parler et travailler avec une équipe d'enseignant chercheur.

A l'issu de cette formation, j'ai pour but de continuer mes études en sécurité informatique afin de pouvoir devenir intervenant extérieur.

Durant ce stage, j'ai du, avec l'aide de mes encadrant, étudier les jeux de Nim dans leur généralité et plus particulièrement le jeu de la foret et sa méthode de résolution propose par Josef Ulehla en 1979.

Dans ce rapport, je vais vous présenter le travail effectué durant ce stage et aussi vous présenter les implantations des différents jeux étudiés.

\subsection{Présentation du département informatique}
\label{sub: Présentation du département informatique}

Le département informatique de l’université de Rouen est découpé en deux équipes de recherches, la première est au technopole du Madrillet. Cette équipe du département est spécialisé dans la recherche algébrique de l'informatique comme la cryptographie, la théorie des graphes et la théorie des automates. Alors que la seconde département est situé a Mont Saint Aignan et est spécialisé dans la recherche en bio-informatique et la théorie du texte.

Le département informatique de la l’université est dirigé par M. Ayoub Otmani compte une trentaine d'enseignant-chercheur permanent et fait appel a une vingtaine d'intervenant professionnel extérieurs. Le département forme aussi des doctorants, a ce jour deux doctorants sont présents, M. Vlad Dragoi et M. Clement Miklarz. M. Ali Chouria a quand a lui validé sa thèse durant la période de ce stage.

Ce stage fait suite a un projet effectué au cours de l’année dans le cadre du cours de théorie des graphes de Madame Selmi au cinquième semestre de licence informatique.