Durant ce stage, j'ai eu l'occasion de pouvoir mettre en pratique la théorie acquise durant mes 3 années de licence.

Ceci m'a permis de voir que même si je pensais que certains cours ne seraient pas utiles car trop théoriques, ils avaient leur importance dans la logique et la gestion d'un problème technique et mathématique.

Ce stage a été aussi pour moi l'occasion d'apprendre un nouveau langage et de pouvoir valider mes connaissances théoriques dans les différents paradigmes de programmation comme la programmation impérative, la programmation orientée objet et la programmation fonctionnelle.

\subsection{Connaissances acquises durant le stage}    
\label{sub:Connaissances acquises durant le stage}
  
Ce stage m'a permis de mieux appréhender la gestion d'un projet. En effet, grâce à lui, j'ai eu l'occasion de pouvoir gérer mon travail seul et de pouvoir demander de l'aide aux moments où j’étais bloqué.

J'ai aussi pu élargir mes connaissances dans le milieu de l'informatique théorique au cours de différents séminaires et présentations au sein du laboratoire informatique.

\subsection{Connaissances sur le m\'etier d'enseignant chercheur} 
\label{sub:Connaissances sur le m\'etier d'enseignant chercheur}

Durant ces deux mois de stage, j'ai pu avoir un aperçu du métier d'enseignant-chercheur. J'ai pu assister à différents événements au sein du laboratoire comme le séminaire de Normastic ou encore la soutenance de thèse de M. Ali Chouria.

Durant ces événements, j'ai pu voir qu'il nous restait encore beaucoup de travail avant de pouvoir être totalement apte à entrer dans le milieu de la recherche. Même si les exposés étaient clair et précis, nous nous sommes vite rendu compte qu'il nous restait encore beaucoup de bases à appréhender.

