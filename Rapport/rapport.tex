\documentclass{univ-projet}

\usepackage[utf8]{inputenc}
\usepackage[T1]{fontenc}
\usepackage[francais]{babel}
\usepackage{hyperref}
\usepackage{ulem}
\usepackage{tikz}
\usepackage{algorithm}
\usepackage{algorithmic}

\usetikzlibrary{arrows,automata}
\usetikzlibrary{mindmap}



\logo{img/logo_univ.png}
\author{Tristan Rodriguez}
\relecteur{Tristan Rodriguez}
\filiere{L3info}
\projet{Stage 2015-2016}
\projdesc{Stage au sein du département informatique du Madrillet}
\title{Rapport de stage}
\version{1.1}
\signataire{Carla Selmi et Jean-philippe Dubernard}

%historique du stage en page de garde
\histentry{0.1} {du 9 au 14 mai} {recherche bibliographique (Théorème de Spague-Grundy et différents jeux de Nim) et étude de l'article de J. Ulehla}
\histentry{0.1.1} {du 16 au 21 mai} {recherche technique sur le python et implantation d'un jeu de Nim ordinaire}
\histentry{0.1.2} {du 23 au 27 mai} {étude du travail de Samuel Giraudo et Cédric Leroy sur les listes préfixes et suffixe dans les arbres}
\histentry{0.2} {du 30 mai au 2 juin} {fin de l'implantation de la méthode Ulehla et une foret géré grâce aux listes préfixe et suffixe}
\histentry{0.3.1} {du 6 au 10 juin} {début de la rédaction du rapport et mise en forme de la preuve sur les listes préfixe et suffixe}
\histentry{0.3.2} {du 12 au 16 juin} {étude du jeu de Chomp revisité}
\histentry{1.0} {du 20 au 30 juin} {rédaction du rapport de stage}
\histentry{1.1} {du 4 au 9 juillet} {finalisation du rapport et rendu du travail}

%Document principal
\begin{document}
  \maketitle
  \vspace*{\stretch{1}}
    \begin{center}
      \textbf{Remerciements}

      Avant toute choses, je tient a remercier Carla Selmi, Jean-phlippe Dubernard qui m'ont donné la possibilité de découvrir le métier d'enseignant chercheur et m'ont fait confiance durant ce stage. Je tient également a remercier Yacine Hmito qui a travaillé avec moi et qui m'a aidé et a partagé son travail et ses connaissances.

      Ce stage a  été proposé et encadré par Carla Selmi et Jean-phlippe Dubernard.

      Ce rapport a été rédigé \LaTeX.
    \end{center}
  \vspace*{\stretch{1}}

  \clearpage
  
  \tableofcontents
  \clearpage

  \section{Introduction}
  \label{sec:Introduction}
  Durant la formation de licence informatique, nous avons obtenu de fortes base théoriques et pratiques et nous avons la possibilité de pouvoir mettre a profit nos connaissances durant un stage de deux mois. Le but de ce dernier est d'avoir une première approche du milieu professionnel dans le monde de l'informatique et de pouvoir évaluer nos connaissances acquises.

c'est dans ce but que j'ai obtenu un stage au sein du département informatique de l’université de Rouen.

Ce stage fut pour moi une bonne occasion de pouvoir découvrir le milieu de la recherche informatique et de pouvoir valider mes choix d'orientation. En effet, ce stage m'a permis de pouvoir parler et travailler avec une équipe d'enseignant chercheur.

A l'issu de cette formation, j'ai pour but de continuer mes études en sécurité informatique afin de pouvoir devenir intervenant extérieur.

Durant ce stage, j'ai du, avec l'aide de mes encadrant, étudier les jeux de Nim dans leur généralité et plus particulièrement le jeu de la foret et sa méthode de résolution propose par Josef Ulehla en 1979.

Dans ce rapport, je vais vous présenter le travail effectué durant ce stage et aussi vous présenter les implantations des différents jeux étudiés.

\subsection{Présentation du département informatique}
\label{sub: Présentation du département informatique}
A COMPLETER
Le département informatique de l’université de Rouen est découpé en deux equipes de recherches, la première est au technopole du Madrillet. Cette équipe du département est spécialisé dans la recherche algébrique de l'informatique comme la cryptographie, la théorie des graphes et la théorie des automates. Alors que la seconde département est situé a Mont Saint Aignan et est spécialisé dans la recherche en bio-informatique et la théorie du texte.

Ce stage fait suite a un projet effectué au cours de l’année dans le cadre du cours de théorie des graphes de Madame Selmi et a été effectué au sein du département du Madrillet dirigé par M. Ayoub Otmani.

  
  \section{Généralités}
  \label{sec:généralités}
  Durant le stage, nous avons commencé par étudier théoriquement le sujet. Comme nous avons étudié l'article de J Ulehla, nous avons commencé par faire une étude bibliographique.

Dans cette partie, nous allons présenter les outils théoriques dans l'ordre d’étude. Ces outils ont été nécessaires pour pouvoir comprendre et mieux implanter le jeu de la foret.

\subsection{Jeu de Nim}
\label{sub:Jeu de Nim}
  Un jeu de  Nim est un jeu de stratégie pure, c'est a dire que ce jeu ne laisse aucune part au hasard et ne permet pas la fin sur une égalité entre les deux joueurs.

  \textit{
    Par exemple, un jeu de Nim avec un tas d'allumettes ou l'on enlève une ou plusieurs allumettes dans ce tas et le dernier ayant retiré la dernière allumette gagne est un jeu de Nim.
  }
  \subsubsection{Jeu de Nim simple}
  \label{subsub: Jeu de Nim simple}
    Un jeu de Nim simple est un jeu de Nim comme défini ci dessus a ceci près qu'il peut être résolu de façon rapide. Ces jeux ont été résolus par Charles Bouton en 1901\cite{jfji} qui a trouvé un algorithme permettant le gain.

    \textit{
      Le graphe suivant permet de représenter tous les cas possibles pour un jeu de Nim décrit ci dessus avec un tas a 3 allumettes
    }
    \begin{figure}[h]
      \centering
        \begin{tikzpicture}[->,>=stealth',shorten >=1pt,auto,node distance=3.5cm, semithick]
          \tikzset{
            state/.style = {shape=rectangle, rounded corners, draw, align=center,
                            top color=white, bottom color=blue!20, text centered}
          }

          \node[initial, state] (3)                    {$3\ allumettes$};
          \node[state]          (f) [below of=3]       {$fin\ du\ jeu$};
          \node[state]          (2) [below right of=f] {$2\ allumettes$};
          \node[state]          (1) [below left of=f]  {$1\ allumette$};

          \path (3) edge (2)
                (3) edge (1)
                (3) edge (f)
                (2) edge (1)
                (2) edge (f)
                (1) edge (f);      
        \end{tikzpicture}
      \caption{graphe du jeu de Nim simple avec un tas a 3 allumettes}
    \end{figure}

    Le principe de l'algorithme de Bouton est de donner une valeur a chacun des états du graphe représentant le jeu appelé \textit{Nimber}. Dans le cas d'un jeu \`a un tas unique, le \textit{nimber} d'un états est la valeur du états elle même.

    L'attribution de ce \textit{nimber} est récursive, tout les états finaux (un seul état final dans le cas d'un jeu a un tas) ont comme \textit{nimber} 0. Ensuite, pour tout états, \textit{nimber} est égal au plus petit entier qui n'est pas égal au nimber de chacun de ces fils. Nous pouvons bien facilement voir alors que ce nimber est une interprétation du noyau d'un graphe. En effet, tous les éléments du noyau du graphe d'un jeu seront les éléments ayant un nimber égal a 0.

    \textit {
    reprenons l'exemple précédent, l’état ou le jeu est fini aura comme valeur de nimber 0, ensuite pour l’état avec 1 allumette, nous aurons comme valeur de nimber 1. Comme nous connaissons le nimber de tous les fils de l’état avec deux allumettes, nous pouvons donner comme valeur de nimber le plus petit entier qui n'est égal a aucune valeur de nimber de ses fils, c'est pourquoi nous lui attribuons la valeur 2. Enfin nous appliquons le même procédé pour l’état avec 3 allumettes et nous lui attribuons la valeur 3.
    }

    Ce \textit{nimber} permet de pouvoir savoir si oui ou non on peut gagner a partir d'une position donné de jeu.

    Si cette valeur est égale a 0, alors on dit que la position est une \textit{position gagnante}, sinon on dit que cette position est une \textit{position perdante}. Si jamais un joueur commence a partir d'une position perdante, alors il possède une stratégie gagnante, en effet lorsque le joueur est dans une position perdante, alors il y a forcement un état successeur de la position dans laquelle se trouve le joueur qui possède une valeur de nimber valant 0 qui permettra au joueur de gagner.

    Dans le cas contraire, la position dans laquelle se trouve le joueur a un nimber valant 0 alors il n'a pas d’état successeur permettant de gagner. En effet comme la valeur de nimber est le plus petit entier n’étant pas égal aux nimber de ses successeurs, alors il n'y a pas de valeur de nimber de ses successeurs égal a 0. Cela ne permet pas de gagner car jamais le joueur ne pourra jamais atteindre l’état final.

  \subsubsection{Jeu de Nim complexe}
  \label{subsub:Jeu de Nim complexe}

    Un jeu de Nim complexe est un ensemble de jeu de Nim simples. Ils ont été résolus indépendamment par Roland Sprague en 1935 et Patrick Grundy en 1939\cite{jfji}.

    \textit{
      Par exemple, un jeu de Nim avec plusieurs tas d'allumettes ou l'on enlève une ou plusieurs allumettes dans un tas a la fois et le dernier joueur ayant retiré la dernière allumette est un jeu de Nim complexe.
    }

    \textit{
      Le graphe suivant est le graphe d'un jeu avec un tas avec 3 allumettes et un tas avec 2 allumettes.
    }

     \begin{figure}[h]
      \centering
        \begin{tikzpicture}[->,>=stealth',shorten >=1pt,auto,node distance=3.5cm, semithick]
          \tikzset{
            state/.style = {shape=rectangle, rounded corners, draw, align=center,
                            top color=white, bottom color=blue!20, text centered}
          }

          \node[initial, state] (3)                    {$3\ allumettes$};
          \node[state]          (f) [below of=3]       {$fin\ du\ jeu$};
          \node[state]          (2) [below right of=f] {$2\ allumettes$};
          \node[state]          (1) [below left of=f]  {$1\ allumette$};

          \node[state]          (ff)[right of=2]       {$fin\ du\ jeu$};
          \node[state]          (11)[above of=ff]      {$1\ allumette$};
          \node[initial, state] (22)[above of=11]      {$2\ allumettes$};

          \path (3) edge (2)
                (3) edge (1)
                (3) edge (f)
                (2) edge (1)
                (2) edge (f)
                (1) edge (f)
                (22) edge (11)
                (11) edge (ff);      
        \end{tikzpicture}
      \caption{graphe du jeu de Nim complexe avec un tas a 3 allumettes et un tas a 2 allumettes}
    \end{figure}

    La résolution de ces jeux est une généralisation de la méthode de résolution des jeux de Nim simples. Le principe est de \textit{diviser pour mieux régner}. En effet, nous divisons le jeu en autant de jeux de Nim simples qu'il y a de tas.

    Comme les jeux de Nim simples, on commence par donner les valeurs de nimber a tous les états du du graphe du jeu, puis pour savoir si une position du jeu est gagnante, nous devons faire une opération spéciale appelé \textit{addition de Nim}.

    Cette opération est une addition bit a bit des nimbers en binaire sans retenue. Ceci revient a faire un \textit{ou exclusif} entre les valeurs des nimbers de tous les tas de l’état du jeu.

    Comme le jeu de Nim simple, lorsque le nimber d'une position est égal a 0, on se trouve dans une position gagnante du jeu et dans tous les autres cas, nous nous trouvons dans une position perdante.

    \textit {
      dans l'exemple suivant, le joueur 1 qui commencera a jouer a partir de la position (3,2), pourra gagner puisqu'il pourra arriver suite a un mouvement a une position gagnante. Pour savoir quel est ce coup gagnant, il suffit de trouver une valeur pour un des deux tas qui permette d'obtenir un nimber a 0. Ce coup est facile a trouver puisqu'il suffit de retirer une allumette dans le tas avec 3 allumettes (2 ou exclusif 2 est bien égal a 0). Le joueur 1 a donc bien une stratégie gagnante.
    }

    Cette méthode de résolution est très utilisé puisque nous pouvons trouver une fonction permettant d'attribuer une valeur de nimber en fonction du jeu que nous étudions.

    C'est ce qu'a fait J. Ulehla en 19xx pour trouver une methode de resolution rapide du jeu de Hackendot de J. von Neumann.

\subsection{Hackendot}
\label{sub:Hackendot}
  Ce jeu, crée par John Von Neumann et John Horton Conway indépendamment, est un jeu de Nim sur les arbres ou l'on retire dans un arbre du jeu le chemin de la racine a un noeud choisi. Le dernier joueur a enlever un nœud du jeu gagne.

  \textit{
    par exemple l'ensemble d'arbre suivant est une situation initial d'un jeu de Hackendot
  }
  \begin{figure}[h]
  \centering
    \begin{tikzpicture}[sibling distance=10em, every node/.style = {shape=rectangle, rounded corners, draw, align=center,
                        top color=white, bottom color=blue!20}, right]

      \node {a}
        child{node {b}}
        child{node {c}};
    \end{tikzpicture}

    \begin{tikzpicture}[sibling distance=10em, every node/.style = {shape=rectangle, rounded corners, draw, align=center,
                        top color=white, bottom color=blue!20}], left]
        \node{d}
        child{node{e}
          child{node{f}}
        }
        child{node{g}};
    \end{tikzpicture}
  \caption{foret d'arbres d'un jeu de hackendot}
  \end{figure}

  La résolution proposé en 1979 par Josef Ulehla est un dérivé de la méthode de résolution des jeux de Nim complexes. En effet, dans son papier \cite{UleHack} Ulehla cherche simplement a chercher une fonction qui permettrai de donner une valeur de nimber a une situation de jeu.

  Dans la suite, nous verrons un arbre comme étant un graphe. Le noyau d'un arbre est construit de la même façon que le noyau d'un graphe, tout élément n'ayant pas de fils appartiendra au noyau et tout élément ayant aucun fils n'appartenant pas au noyau appartiendra au noyau.

  Dans son papier, Ulehla défini plusieurs fonctions :
  \begin{enumerate}
    \item \textit{\texttt{l(f)}} qui permet de \textit{dénoyauter} c'est a dire enlever le noyau du graphe du jeu. $l^n(f)$ est la fonction \textit{\texttt{l}} appliqué \textit{n} fois a la foret \textit{f}.
    \item\textit{\texttt{rip(f)}} (pour Root-ImParity) qui est égal a 0 (resp. 1) si le nombre de racines blanche de \textit{f} est pair (resp. impair).
  \end{enumerate}

  A l'aide de ces deux fonctions, on peut maintenant définir la méthode de résolution du jeu. En effet, pour savoir si on a bien une stratégie gagnante, on commence par \textit{colorer} la foret du jeu. Colorer signifie que tous les nœuds appartenant au noyau sont \textit{blanc} et tous les autres sont \textit{noir}. Ensuite, on \textit{dénoyaute} la foret ce qui permet d'obtenir une nouvelle foret \textit{f'}. Nous répétons ce procédé jusqu’à obtenir une foret $f^n$ vide. La suite des forets $f^n$ permet de savoir si il y a une stratégie gagnante pour la foret du jeu. En effet, lorsqu'il y a au moins un $rip(f^i)$ avec \textit{i} compris entre \textit{1} et \textit{n} vaut 1, alors il y a une stratégie gagnante.

  Dans cette méthode, la fonction pour donner une valeur de nimber a une position de jeu est la fonction \textit{rip} et l'addition de Nim est la suite des valeurs de \textit{rip} de toutes les forets $f^n$.

  Enfin, dans sa méthode, Ulehla permet de savoir quel est le coup gagnant. Comme dans la résolution des jeux de Nim complexe, pour trouver quel est le coup gagnant, il suffit de trouver le coup permettant d'avoir une foret \textit{f''} qui lorsqu'on applique le procédé décrit précédemment, on obtienne une suite de \textit{rip} égaux a 0.

  Pour trouver ce coup gagnant lorsqu'il existe, il faut prendre la dernière foret de la suite noté $f^k$ tel \textit{rip($f^k$) = 1}. A partir de cette foret, on prend l'ensemble des racines blanches et leurs successeurs directs et on regarde quel suppression permet d'obtenir \textit{rip($f^k$) = 0}. On remonte la suite des forets en prenant a chaque fois l'ensemble constitué du coup gagnant de la foret précédente et ses successeurs.

  \textit{
    reprenons l'exemple du début, la première question que nous nous posons est es ce que l'on a une stratégie gagnante en jouant?
  }

  \begin{figure}[h]
  \centering
    \begin{tikzpicture}[sibling distance=10em, every node/.style = {shape=rectangle, rounded corners, draw, align=center,
                        top color=white, bottom color=blue!20}, right]

      \node {a noir}
        child{node {b blanc}}
        child{node {c blanc}};
    \end{tikzpicture}

    \begin{tikzpicture}[sibling distance=10em, every node/.style = {shape=rectangle, rounded corners, draw, align=center,
                        top color=white, bottom color=blue!20}], left]
        \node{d noir}
        child{node{e noir}
          child{node{f blanc}}
        }
        child{node{g blanc}};
    \end{tikzpicture}

    F rip(F) = 0

    \begin{tikzpicture}[sibling distance=10em, every node/.style = {shape=rectangle, rounded corners, draw, align=center,
                        top color=white, bottom color=blue!20}, right]

      \node {a blanc};
    \end{tikzpicture}

    \begin{tikzpicture}[sibling distance=10em, every node/.style = {shape=rectangle, rounded corners, draw, align=center,
                        top color=white, bottom color=blue!20}], left]
        \node{d noir}
        child{node{e blanc}};
    \end{tikzpicture}
    
    l(F) rip(l(F)) = 1

    \begin{tikzpicture}[sibling distance=10em, every node/.style = {shape=rectangle, rounded corners, draw, align=center,
                        top color=white, bottom color=blue!20}, right]

      \node {d blanc};
    \end{tikzpicture}
    
    $l^2(F) rip(l^2(F)) = 1$
    
    \begin{tikzpicture}[sibling distance=10em, every node/.style = {shape=rectangle, rounded corners, draw, align=center,
                        top color=white, bottom color=blue!20}, right]
      \node {};
    \end{tikzpicture}
    
    $l^3(F) rip(l^3(F)) = 0$
  \caption{Suite des forets dénoyautés}
  \end{figure}

  \textit{
    Nous savons qu'il existe un coup gagnant en faisant la suite des forets dénoyautés et que nous devons chercher dans $l^2(F)$. Nous savons que dans cette foret, il suffit de retirer le nœud \texttt{d} qui permet d'obtenir rip égal a 0.
  }

  \textit{
    Nous cherchons maintenant dans la foret l(F) quel est le coup permettant d'obtenir un rip égal a 0. Nous savons que dans la foret précédente, le coup gagnant était \texttt{d} donc ici, nous regardons l'ensemble constitué de \texttt{d, e}.
    Nous voyons que c'est le nœud \texttt{d} qu'il faut supprimer pour avoir un rip égal a 0.
  }

  \textit{
    Enfin, nous appliquons le même procédé a la foret F et nous trouvons que le coup gagnant est \texttt{g}.
  }
\subsection{Parcours sur les arbres}
\label{sub:Parcours sur les arbres}
\hypertarget{GenArbres}
  Durant l'implantation du jeu de Hackendot, nous nous sommes retrouvé face a une difficulté, comment implanter des arbres quelconques en python?

  Pour répondre a ce problème, madame Selmi m'a fourni le travail d'un ancien étudiant qui a utilisé uniquement le parcours préfixe et suffixe sur les arbres et a permis de gérer l’intégralité de ce problème.

  Quelque soit le parcours utilisé, nous regardons de gauche a droite les nœuds dans l'arbre, c'est a dire que lorsque nous sommes sur un nœud donné, nous regardons cherchons d'abord a découvrir récursivement tout son fils gauche puis nous allons découvrir sont fils droit.

  Tout d'abord, nous avons utilisé le parcours préfixe sur une foret. Ce parcours est effectué dans l'ordre de \textit{découverte} des nœuds dans la foret. Nous pouvons obtenir la liste préfixe, noté P, d'un arbre en mettant les nœuds dans l'ordre de découverte d’après le parcours préfixe.

  \textit{ 
    par exemple, dans l'arborescence suivante, la liste préfixe est (1, 2, 3, 4)
  }

  \begin{figure}[h]
    \centering
    \begin{tikzpicture}[sibling distance=10em, every node/.style = {shape=rectangle, rounded corners, draw, align=center,
                        top color=white, bottom color=blue!20}], left]
        \node{1}
        child{node{3}
          child{node{4}}
        }
        child{node{2}};
    \end{tikzpicture}
    \caption{arborescence de liste préfixe (1, 2, 3, 4)}
  \end{figure}

  Ensuite, nous avons utilisé le parcours suffixe sur une foret. Ce parcours est effectué dans l'ordre de fin de découverte des nœuds dans la foret. Un nœud est entièrement découvert lorsque nous avons entièrement exploré tous ses successeurs. Grâce a ce parcours, nous pouvons créer la liste suffixe, noté S, en mettant dans l'ordre de découverte par rapport a ce parcours les nœuds de la foret.

  \textit{
    dans l'exemple précédent, la liste suffixe de l'arborescence est (2, 4, 3, 1)
  }

  Nous avons aussi défini les listes \textit{S(n)} et $P^{-1}(n)$ comme étant respectivement la liste suffixe a partir d'un nœud n et la liste des nœuds du début de la liste préfixe jusqu'au nœud n.

  \textit{
    dans l'exemple précédent, prenons le nœud 2, alors la liste S(2) est la liste (4, 3, 1) et la liste $P^{-1}(2)$ est la liste (1).
  }

  Grâce a ces différentes listes, nous avons pu connaitre le père direct d'un nœud \textit{n} de la foret. En effet, nous remarquons que le père direct du nœud est le premier élément de \textit{S(n)} commun a la liste $P^{-1}(n)$ .

  Nous avons aussi remarqué que tous les nœuds du chemin de la racine a un nœud \textit{n} était tous les nœuds commun a la liste S(n).

  Durant ce stage, la preuve de ces deux éléments a été fourni.

\subsection{Jeu de Chomp}
\label{sub:Jeu de Chomp}

Le dernier jeu que nous avons étudié est le jeu de Chomp revisité.Comme le chomp, nous jouons sur une tablette carré ou rectangle et nous avons cependant modifié les règles pour faciliter le jeu, nous devons choisir une case et supprimer soit la ligne soit la colonne de la case choisie. Le dernier joueur a enlever une case est le vainqueur.

\textit{
  voici une situation de jeu avec une tablette rectangle 3x4
}

\begin{figure}[h]
  \centering
  \begin{tikzpicture}
    %carre principal
    \fill[ top color=white, bottom color=blue!20] (0,0) -- (0, 1.5) -- (2,1.5) -- (2,0) --(0,0);
    \draw (0,0) -- (0, 1.5) -- (2,1.5) -- (2,0) --(0,0);
    %lignes
    \draw (0,0.5) -- (2,0.5);
    \draw (0,1) -- (2,1);
    %colonnes
    \draw (0.5, 0) -- (0.5, 1.5);
    \draw (1, 0) -- (1, 1.5);
    \draw (1.5, 0) -- (1.5, 1.5);
  \end{tikzpicture}
  \caption{situation de jeu du Chomp}
 
\end{figure}
Au démarrage, ce jeu devait être sur les polyominos convexe (sans trou) cependant, nous nous sommes rendus compte que travailler sur une tablette ou sur un polyomino revenait au même. En effet, nous pouvons voir facilement qu'un polyomino convexe est compris dans une tablette, c'est pourquoi, nous avons généralisé directement sur une tablette.

Ensuite, nous avons cherché a diviser ce jeu de façon a pouvoir utiliser le théorème de Sprague-Grundy, nous avons donc pensé que le but calculer le nombre de ligne et de colonnes. Nous avons pensé que si le nombre de ligne ou le nombre de colonnes était impair, le coup était gagnant.

Puis nous avons implanté le jeu avec cette idée, cependant nous sommes tombé face a un contre exemple. C'est pourquoi nous avons essayé de faire en sorte de revoir la façon de calculer si le coup est gagnant. Nous avons donc pensé que lorsque nous avions plusieurs composantes connexes, nous pouvions les mettre sous la forme d'une tablette avec comme nombre de ligne la somme du nombre de ligne de toutes les composantes connexes et de la même façon le nombre de colonnes la somme des colonnes de toutes les composantes connexes.

A la fin de ce stage, nous n'avons pas réussi a savoir comment faire pour savoir si il existe une stratégie gagnante a partir d'une position de jeu donné et si oui comment calculer le coup gagnant.

Cependant, nous avons conjecturé que pour une position donnée le nimber valait :
\begin{itemize}
  \item \texttt{m} dans le cas d'une tablette 1xm ou mx1 (prouvé)
  \item \texttt{0} si la tablette est de la forme 2nx2m (prouvé)
  \item \texttt{1} si dans une tablette nxm avec m ou n impair m + n est impair
  \item \texttt{2} sinon
\end{itemize}
  Nous avons mis au point un petit programme permettant la vérification de cette conjecture et a ce jour, nous n'avons pas trouvé de contre exemple.
    
  \section{Travail effectué}
  \label{sec:Travail effectué}
  Suite a l’étude théorique des jeux de Nim et de leurs méthodes de résolutions, il a été temps de les implanter.

Dans ce but, j'ai décide d'apprendre une nouveau langage, le \textit{python}, qui est un langage fortement typé, interprété et est aussi orienté objet. 

Ce langage est de plus en plus présent dans le milieu professionnel et est très utilisé dans le milieu de la sécurité informatique. C'est pour ces raisons que j'ai pensé que ce stage était l'occasion de pouvoir apprendre ce nouveau langage.

toutes les implantations effectué durant ce stage sont disponible sur GitHub \footnote{\url{https://github.com/wampixel/nim-games}} en téléchargement libre.
\subsection{Jeu de Nim simple}
\label{sub:Jeu de Nim simple}

La première implantation effectué était principalement pour apprendre le python.

En effet, le jeu de Nim simple était simple a programmer car il ne demande que très peut de mémoire et la méthode de résolution est basé sur l’opération XOR\footnote{ou exclusif} sur l'ensemble des entiers.

Pour implanter ce jeu, nous avons eu besoin de deux fichiers, le premier\texttt{ia.py} qui permet de pouvoir calculer si il existe un coup gagnant et savoir quel est ce coup, et le second \texttt{nim.py} qui est une implantation du jeu de Nim complexe a plusieurs tas.

Pour pouvoir implanter la methode de resolution, nous avons eu besoin de plusieurs fonction :
\begin{itemize}
  \item \texttt{getwinningStrat(l)} qui retourne le coup a jouer pour gagner si il existe
  \item \texttt{numAdd(l)} qui retourne le nimber d'une situation de jeu 
\end{itemize}

L'algorithme pour avoir le coup gagnant d'une situation de jeu est la suivante :
\begin{algorithm}[hbt]
  \caption{calcul le coup gagnant si il existe}
  \begin{algorithmic}
    \REQUIRE l une liste de tas représentant une situation de jeu
    \ENSURE un tuple de la forme (t, n) avec t le tas ou jouer, n le nombre a enlever
    
    \STATE $s \leftarrow numAdd(l)$
    \IF{$s = 0$}
      \RETURN $(0, 1)$
    \ELSE
      \STATE $i \leftarrow 0$
      \WHILE{$i < len(l)$}
        \STATE $f \leftarrow f\ xor\ l[i]$
        \IF{$f = 0$}
          \RETURN $(i, l[i])$
        \ELSE
          \STATE $k \leftarrow 0$
          \WHILE{$k < l[i]$}
            \IF{$f\ xor\ k = 0$}
              \RETURN $(i, (l[i] - k))$
            \ELSE
              \STATE $k \leftarrow k + 1$
            \ENDIF
          \ENDWHILE
        \ENDIF
        \STATE $i \leftarrow i + 1$
      \ENDWHILE
    \ENDIF
  \end{algorithmic}
\end{algorithm}    

Le but de cet algorithme est de simplement tester si on est dans une position gagnante au démarrage (numAdd(l) = 0). Si c'est le cas, comme présenté dans les généralités, on ne peux pas gagner, c'est pourquoi on joue un coup dont on est sur de l'existence (tas 0, 1 allumette).

Si jamais on n'est pas dans une position gagnante, alors on est sur et certain qu'il existe un coup gagnant, c'est pourquoi on commence a calculer ce coup.

On commence a tester si en enlevant chaque tas complétement et on regarde l'addition de Nim pour ce fils. Si jamais on a un résultat de 0 pour l'addition de Nim, alors on a trouve le coup gagnant et on le retourne. Sinon on va commencer a enlever les allumettes une a une dans chaque tas et on trouvera forcement le coup gagnant que l'on retournera.

Cet algorithme est en temps linéaire sur la somme des éléments de la liste. Dans le pire des cas, on aura du calculer tous les coups possibles pour trouver le coup gagnant.

Cet algorithme repose sur une addition spéciale appelé addition de Nim. Elle a été implanté dans la fonction \texttt{numAdd} et fonctionne de la façon suivante :

\begin{algorithm}[hbt]
  \caption{addition de Nim}
  \begin{algorithmic}[h]
    \REQUIRE l une liste de tas représentant une situation de jeu
    \ENSURE le nimber de la situation l
    \STATE $i \leftarrow 0$
    \STATE $s \leftarrow 0$
    \FORALL{$i\ valeurs\ de\ la\ liste\ l$}
      \STATE $s \leftarrow s\ xor\ i$
    \ENDFOR
    \RETURN $s$
  \end{algorithmic}
\end{algorithm}

Cette fonction calcul simplement une addition sans retenue entre toutes les valeurs des tas en base 2. Ceci revient a faire un XOR (ou exclusif) entre ces valeurs. Cette fonction est en temps linéaire par rapport a la longueur de la liste l.

Ces algorithmes permettent d'implanter le théorème de Sprague-Grundy. Suite a ceci, nous avons cherché a utiliser ce théorème dans plusieurs autres jeux pour vérifier si nous pouvions a une méthode de calcul de nimber près toujours résoudre les jeux.

\subsection{Hackendot}
\label{sub:Hackendot}

Après avoir réussi a implanter le jeu de Nim et le théorème de Sprague-Grundy, nous avons attaqué le travail principal de ce stage. Nous avons donc étudié la méthode de résolution du jeu de Hackendot proposé par J. Ulehla.

Nous avons décidé d'utiliser ce coup ci la partie orienté objet du python. Ceci m'a permis de pouvoir apprendre a utiliser les objets en python.

La première classe que nous avons crée est la classe \texttt{forest}, celle ci permet de gérer les forets et nous ne détaillerons pas ici le contenu de classe car c'est une gestion de listes préfixe et suffixe simple avec les éléments démontrées dans les \hyperlink{GenArbres}{généralités}

Nous avons ensuite crée une classe \texttt{iaUlehla} permettant de créer une intelligence artificielle simplifié pour calculer le coup gagnant dans ce jeux.

Dans cette classe, il y a les methodes :
\begin{itemize}
  \item \texttt{getWinningStrat(forest)} qui permet de connaitre le coup gagnant pour la foret \textit{forest}.
  \item \texttt{lfunction(forest)} qui permet d'effectuer la fonction l sur la foret \textit{forest}.
  \item \texttt{ripFunction(forest)} qui permet d'effectuer la fonction rip sur la foret \textit{forest}.
  \item \texttt{getNodeForWin(forestL)} qui permet d'implanter le raisonnement de Ulehla pour trouver le coup gagnant a partir de la liste des dénoyautages successifs de la foret initiale \textit{forestL}.
\end{itemize}

Cette classe utilise aussi deux méthodes privées qui permettent de \textit{colorer} la foret.

La première \texttt{colorNodeSucc(succList)} qui permet de colorer un nœud grâce a la liste de ses successeurs. Cette prend en entrée une liste de couleurs qui représente la liste des couleurs de ses successeurs et elle retourne \texttt{BLANC} si et seulement si toutes les couleurs de la liste sont a \texttt{NOIR}. Si jamais la couleur d'au moi ns un des successeurs est a \texttt{BLANC}, la méthode retourne \texttt{NOIR}.

La seconde méthode privée \texttt{colorNode(forest, node)} elle permet de colorer le noeud \texttt{node} dans la foret \texttt{forest}.Cette methode construit recursivement la liste des couleurs des successeurs du noeud \texttt{node} et utilise la methode \texttt{colorNodeSucc(succList)}.

L'avantage de cette methode est de ne pas colorer entierement la foret, en effet, pour savoir si un noeud est blanc, nous devons etre sur que tous les fils du noeud que l'on veut colorer sont bien \texttt{NOIR}, cependant, pour qu'un noeud soit \texttt{NOIR}, il suffit de savoir qu'un fils est bien \texttt{BLANC}.

Dans la suite, nous allons presenter les algorithmes permettant de savoir si un coup gagnant existe, puis nous presenterons ensuite les algorithmes des fonctions propre a la methode Ulehla (\texttt{ripFunction} et \texttt{lFunction})

\begin{algorithm}[hbt]
  \caption{Calcul si le coup gagnant existe}
  \begin{algorithmic}
    \REQUIRE f une foret quelconque
    \ENSURE le noeud a supprimer pour entrer dans une strategie gagnante si c'est possible, le premier noeud de la liste sinon
    \STATE $forestL \leftarrow [f]$
    \WHILE{$f\ is\ not\ empty$}
      \STATE $f \leftarrow lfunction(f)$
      \STATE $forestL \leftarrow forestL.append(f)$
    \ENDWHILE
    \STATE $i \leftarrow -1$
    \STATE $k \leftarrow len(forestL) - 1$
    \WHILE{$k > 0\ and\ i = -1$}
      \IF{$ripFunction(forestL[k]) = 1$}
        \STATE $i \leftarrow k$
      \ENDIF
      \STATE $k \leftarrow k - 1$
    \ENDWHILE
    \IF{$i = -1$}
      \RETURN $f.getPrefixeList()[0]$
    \ELSE
      \RETURN $getNodeForWin(forestL[0...i + 1])$
    \ENDIF
  \end{algorithmic}
\end{algorithm}

Cet algorithme a besoin de la fonction \texttt{getNodeForWin(forestL)} qui implante l'algorithme suivant :

\begin{algorithm}[hbt]
  \caption{calcul le coup gagnant dans la suite des forets dénoyautées}
  \begin{algorithmic}
    \REQUIRE forestL la suite des forets dénoyautées
    \ENSURE le noeud a supprimer pour entrer dans une strategie gagnante
    \STATE $f \leftarrow forestL[len(forestL) - 1]$
    \STATE $tmp \leftarrow f.getRoots()$
    \STATE $root \leftarrow []$
    \FORALL{$r\ in\ tmp$}
        \IF{$colorNode(f, r) = BLANC$}
          \STATE $root \leftarrow root.append(r)$
        \ENDIF
    \ENDFOR
    \STATE $i \leftarrow len(forestL) - 1$
    \STATE $possibility \leftarrow root$

    \WHILE{$i >= 0$}
      \STATE $f \leftarrow forestL[i].getForest()$
      \STATE $tmp \leftarrow possibility$

      \FORALL{$p\ in\ possibility$}
        \STATE $tmp \leftarrow tmp.extend(f.getSuccNode(p))$
      \ENDFOR

      \STATE $j \leftarrow 0$
      \STATE $b \leftarrow TRUE$

      \WHILE{$j < len(tmp)\ and\ b$}
        \STATE $f \leftarrow f.delNodeToRoot(tmp[j])$
        \IF{$ripFunction(f) = 0$}
          \STATE $b \leftarrow FALSE$
          \STATE $tmp \leftarrow [tmp[j]]$
        \ENDIF
        \STATE $j \leftarrow j + 1$
      \ENDWHILE
      \STATE $possibility \leftarrow tmp$
    \ENDWHILE
  \end{algorithmic}
\end{algorithm}

Cet algorithme est linaire sur le nombre de fils des coups possibles. En effet, nous pouvons voir que au premier tour de boucle, nous devons trouver tous les fils des coups dit possible (c'est a dire les racines blanches de la dernière foret), puis nous devons parcourir cette liste de successeurs pour trouver le coup qui permet de donner une suite de forets avec un rip a 0.

Ceci entraine que l'algorithme pour savoir si un coup gagnant existe est quadratique au pire des cas car on doit parcourir la liste complétement une fois pour trouver quelle est la dernière foret avec un rip égal a 1, puis nous devons parcourir tous les fils possibles pour chaque coup gagnant possibles pour trouver le coup gagnant. Si jamais ce coup gagnant n'existe pas, cet algorithme est linéaire sur la longueur de la liste des forets dénoyautées.

les fonctions précédentes permettent a partir du nimber d'un position de savoir ou est le coup gagnant. Ce nimber est attribué a une situation grâce a la fonction \texttt{rip} et la fonction \texttt{l} présenté dans l'article de Ulehla.

L'algorithme de ces fonctions sont les suivantes :

\begin{algorithm}[hbt]
  \caption{fonction rip}
    \begin{algorithmic}
    \REQUIRE f une foret
    \ENSURE 1 si si un nombre de racines blanches impair, 0 sinon
    \STATE $nbRW \leftarrow 0$
    \FORALL{i in f.getPrefixList()}
      \IF {$i\ est\ une\ racine\ et\ colorNode(f, i) = BLANC$}
        \STATE $nbRW \leftarrow nbRW + 1$
      \ENDIF
    \ENDFOR
    \RETURN ($nbRW modulo 2$)
  \end{algorithmic}
\end{algorithm}    

Cet algorithme est linéaire sur le nombre de nœud de la foret. En effet, pour trouver le nombre de racines blanches, il faut parcourir la foret complète et trouver tous les nœuds sans prédécesseurs(les racines) et trouver la couleur du nœud.

\begin{algorithm}[hbt]
  \caption{fonction l}
  \begin{algorithmic}
    \REQUIRE f une foret
    \ENSURE f dénoyauté
    \STATE $whiteNode \leftarrow []$
    \FORALL{i in f.getPrefixList()}
      \IF{$colorNode(f, i) = BLANC$}
        \STATE $whiteNode \leftarrow whiteNode.append(i)$
      \ENDIF
    \ENDFOR
    \FORALL{i in whiteNode}
      \STATE $f \leftarrow f.delNode(i)$
    \ENDFOR
  \end{algorithmic}
\end{algorithm}

Cet algorithme est linéaire sur le nombre de nœud de la foret. En effet, nous devons parcourir toute la foret pour trouver les nœuds blancs et les supprimer c'est immédiat(nous gérons les forets avec des listes).

Une fois ce travail terminé et mes encadrant étant content de mon travail, madame Selmi m'a proposé une recherche nouvelle, le jeu de Chomp revisité. Cette recherche est un travail non effectué, nous avons donc dus chercher comment le résoudre et nous avons cherché a appliquer le théorème de Sprague-Grundy.

\subsection{Jeu de Chomp}
\label{sub:Jeu de Chomp}

Avant tout, ce jeu n'existait pas a notre connaissance, c'est pourquoi nous avons dus chercher comment faire pour pouvoir appliquer le théorème de Sprague-Grundy. Nous avons conjecturé plusieurs résultats et prouvé les plus simples. Cependant, nous n'avons a ce jour toujours pas trouvé comment faire pour trouver le coup gagnant et gagner a tous les coup.

Nous avons quand même implante le jeu et la solution que nous avions trouvé. 

Cette implantation est faite en deux fichiers :
\begin{itemize}
  \item \texttt{chomp.py} qui est le jeu en lui même
  \item \texttt{resolver.py} qui est l'implantation de la methode de résolution conjecturé.
\end{itemize}

Dans ce jeu, nous jouons, comme le chomp standard, sur une tablette. Nous choisissons un carré et nous supprimons soit la ligne soit la colonne. 

La méthode conjecturée était que lorsque nous avions un nombre de colonnes ou de lignes impair, nous avions un nimber a 1 et un nimber a 0 sinon.

Cependant, nous nous sommes rendus compte que dans certains cas faciles, la méthode de résolution ne prenais pas en compte les composantes connexes identiques.

Nous avons donc cherché a redéfinir notre fonction pour attribuer le nimber et nous avons conjecturé les éléments suivants:
\begin{itemize}
  \item le nimber d'une tablette 1xm ou mx1 est egal a m (prouvé)
  \item le nimber d'une tablette 2mx2n est egal a 0 (prouvé)
  \item le nimber d'une tablette nxm avec n ou m impair est egal a 1 si m+n est impair et 2 sinon (non prouvé)
\end{itemize}

La preuve de ce dernier point n'est pas évidente comme les deux premiers et nous avons travaillé avec madame Selmi sans arriver a voir comment nous y prendre.

  \section{Apports du stage}
  \label{Apports du stage}
  \subsection{Connaissances acquises durant le stage}
    
    \label{sub:Connaissances acquises durant le stage}
    \subsection{Connaissances sur le m\'etier d'enseignant chercheur} 
    \label{sub:Connaissances sur le m\'etier d'enseignant chercheur}
  
  \section{Conclusion}
  \label{sec:conclusion}
  Durant ce stage, j'ai appris beaucoup sur le métier d'enseignant-chercheur et sur la recherche. Ce stage m'a aussi permis de pouvoir mieux appréhender mes inquiétudes sur mes choix d'orientation et de pouvoir être plus sûr de mon envie d'aller en sécurité informatique.

Je pense que ce stage m'a permis de voir un nouveau milieu professionnel que je ne connaissais pas et grâce à mes encadrants, les enseignants-chercheurs du département et mon collègue Yacine Hmito, j'ai pu faire un stage sans être perdu et ils m'ont beaucoup aidé et proposé des solutions qui m'ont permis d'avancer sans me perdre dans le travail.

  \clearpage

  \nocite{*}
  \bibliographystyle{alpha}
  \bibliography{rapport}
  \addcontentsline{toc}{section}{Références}
  
\end{document}
  