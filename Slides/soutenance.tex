\documentclass{beamer}

\usepackage[utf8]{inputenc}
\usepackage[T1]{fontenc}
\usepackage[francais]{babel}
\usepackage{tikz}
\usepackage{graphicx}

\usetikzlibrary{arrows,automata}
\usetikzlibrary{mindmap}

\graphicspath{{img/}}

\usetheme{Singapore}
% \usetheme{Warsaw}

\title{Stage de Licence 3 au sein du département informatique de l’université de Rouen}

\author[Tristan Rodriguez]{Tristan Rodriguez}
\institute{Université de Rouen, L3 Info}
\date{31/08/2016}

\setbeamersize{text margin left=0.5cm, text margin right=0.5cm}
\setbeamertemplate{blocks}[rounded][shadow=true]
\setbeamertemplate{itemize item}[triangle]
\beamertemplatetransparentcovered
\addtobeamertemplate{footline}{\insertframenumber/\inserttotalframenumber}

\begin{document}
  \section{Introduction}
  \label{sec:Introduction}


  % Slide 1 : Titre (présentation des conditions d'acquisition du stage et de Madame Selmi et Monsieur Dubernard)
  \begin{frame}
    \titlepage
  \end{frame}

  % Slide 2 : Table des matières (présentation rapide du travail effectué)
  \begin{frame}
    \frametitle{Plan de la présentation}
    \tableofcontents
  \end{frame}

  %Slide 3 : Présentation du département info
  \begin{frame}
    \frametitle{Présentation du département informatique}
    
    \begin{center}
      \includegraphics[width=0.3\textwidth, height=0.1\textwidth]{logo_univ.png}  
    \end{center}

    \begin{block} {Effectif du département}
      \begin{itemize}
        \item une trentaine d'enseignants-chercheurs
        \item une vingtaine d'intervenants professionnels
        \item dix salles d'ordinateurs en réseau gérées par Monsieur Macadré
      \end{itemize}
    \end{block}
    
    \pause

    \begin{block} {Les thèmes de recherche du département}
      \begin{itemize}
        \item la combinatoire algébrique, énumerative et des mots
        \item la théorie des langages
        \item les automates finis
        \item le calcul symbolique
      \end{itemize}
    \end{block}
  \end{frame}

  % Slide 4 : Présentation des formations du département
  \begin{frame}
    \frametitle{Présentation des formations du département informatique}

    \begin{block} {Les formations principales}
      \begin{itemize}
        \item une licence informatique
        \item un master spécialisé sur le développement logiciel (GIL)
        \item un master spécialisé sur la sécurité informatique (SSI)
        \item un master spécialisé sur la recherche informatique (ITA)
        \item des encadrement pour le doctorat (3 bourses obtenues pour l’année 2016-2017)
      \end{itemize}
    \end{block}
  \end{frame}
  % Slide 5 : Présentation des caractéristiques des jeux de Nim et de plusieurs jeux de nims connus
  \begin{frame}
    \frametitle{Un jeu de Nim, qu'est-ce que c'est ?}

    \begin{block} {C'est un jeu qui :}
      \begin{itemize}
        \item se joue à deux joueurs 
        \item ne fait pas intervenir le hasard
        \item ne peut pas terminer sur une égalité
      \end{itemize}
    \end{block}
    
    \pause

    \begin{block} {Jeux de Nim connus}
      \begin{itemize}
        \item jeu du tas d’allumettes résolu par Charles Bouton en 1901
        \item jeu des tas d'allumettes résolu indépendamment par Roland Sprague en 1935 et Patrick Grundy en 1939
      \end{itemize}
    \end{block}
  \end{frame}

  \section{Travail théorique}
  \label{sec: Travail théorique}

  %Slide 6 : Présentation de la méthode de résolution des jeux de Nim simples
  \begin{frame}
    \frametitle{Résolution du jeu d'un tas de 3 d'allumettes (Bouton)}
    \only<1> {Commençons par dessiner le graphe du jeu}
    \only<2> {Donnons un \textit{nimber} à chaque nœud ;}
    \only<2> {\alert{le nimber d'une position n'est autre que le nombre d'allumettes à cette position.}}
    \only<3> {Comme le nimber de la situation de départ n'est pas nul, il existe une stratégie gagnante à partir de ce nœud.}
    \only<4> {Nous cherchons maintenant comment obtenir un nimber nul à partir de cette position.}
    \only<5> {Dans cette situation simple, pour être sûr de gagner il faut donc retirer toutes les allumettes dès le début du jeu.}

    \only<1-4> {
      \begin{figure}[h]
        \centering
        \begin{tikzpicture}[->,>=stealth',shorten >=1pt,auto,node distance=3.5cm, semithick]
          \tikzset{
            state/.style = {shape=rectangle, rounded corners, draw, align=center, top color=white, bottom color=blue!20, text centered}
          }

          \node[initial, state] (3)                    {$3\ allumettes\ \only<2-3>{\color{red}3}$};
          \node[state]          (f) [below of=3]       {$fin\ du\ jeu \ \only<2>{\color{red}0}\only<4>{\color{red}0}$};
          \node[state]          (2) [below right of=f] {$2\ allumettes\ \only<2>{\color{red}2}$};
          \node[state]          (1) [below left of=f]  {$1\ allumette\ \only<2>{\color{red}1}$};
          
          \path (3) edge (2)
                (3) edge (1)
                (3) edge (f)
                (2) edge (1)
                (2) edge (f)
                (1) edge (f);    
          \only<4->{\color{red} \path (3) edge (f);}  
        \end{tikzpicture}
      \end{figure}
    }
  \end{frame}

  % Slide 7 : Présentation du théorème de Sprague-Grundy
  \begin{frame}
  \frametitle{Résolution d'un jeu de Nim par le théorème de Sprague-Grundy}
    \only<1> {La méthode repose sur le travail de Bouton : prenons un jeu avec deux tas dont chacun a 2 allumettes.}

    \only<2> {Comme on ne peut retirer que dans un tas à la fois, le graphe devient plus grand.}
    \only<3> {On attribue à nouveau les nimbers comme la solution de Bouton.}
    \only<4> {Comme le nimber de la situation de départ est nul, il n'existe pas de stratégie gagnante pour ce joueur.}
    \only<5> {Cette solution nous montre que si l'on joue face à quelqu'un qui connait la méthode pour gagner, nous perdrons à tous les coups si nous jouons en premier.}
    \only<2-4> {
      \begin{figure}[h]
        \centering
        \begin{tikzpicture}[->,>=stealth',shorten >=1pt,auto,node distance=2.5cm, semithick]
          \tikzset{
            state/.style = {shape=rectangle, rounded corners, draw, align=center, top color=white, bottom color=blue!20, text centered}
          }

          \node[initial, state] (22)                     {$2,2\ \only<3-4>{\color{red}0}$};   
          \node[state]          (21) [below left of=22]  {$2,1\ \only<3>{\color{red}3}$};
          \node[state]          (11) [below left of=21]  {$1,1\ \only<3>{\color{red}0}$};
          \node[state]          (2)  [below right of=22] {$2\ \only<3>{\color{red}2}$};
          \node[state]          (1)  [below of=21]       {$1\ \only<3>{\color{red}1}$};
          \node[state]          (f)  [below of=2]        {$final\ \only<3>{\color{red}0}$};
          
          \path (22) edge (21)
                (22) edge (2)
                (21) edge (2)
                (21) edge (1)
                (21) edge (11)
                (11) edge (1)
                (2) edge (1)
                (2) edge (f)
                (1) edge (f);        
        \end{tikzpicture}
      \end{figure}
    }


    \only<3> {Le nimber d'une position est égal au \textit{ou exclusif} du nombre d'allumettes de chaque tas à cette position.}
  \end{frame}

  % Slide 8 : Présentation du jeu de Hackendot et bref historique de travail dessus
  \begin{frame}
    \frametitle{Le Hackendot qu'est-ce que c'est?}
    \only<1> {
      \begin{block}{Quelques informations}  
        \begin{itemize}
          \item un jeu qui se joue sur une forêt (un ensemble d'arbres)
          \item la solution de ce jeu a été proposée par J. Ulehla en 1979
          \item la solution reprend l'idée du théorème de Sprague-Grundy appliqué au jeu de la forêt
        \end{itemize}
      \end{block}
    }

    \only<2-> {
      \begin{block}{Les règles du jeu}  
        \begin{itemize}
          \pause
          \item on choisi un arbre présent dans la forêt du jeu
          \pause
          \item on choisi un nœud dans l'arbre précédemment choisi
          \pause
          \item on retire tout le chemin menant du nœud choisi à la racine de l'arbre choisi 
        \end{itemize} 
      \end{block}
    }

    \only<5-> {
      exemple :
      \begin{figure}[h]
        \centering
        \begin{tikzpicture}[sibling distance=5em, every node/.style = {shape=rectangle, rounded corners, draw, align=center,
                      top color=white, bottom color=blue!20}, right]

          \node {a}
            child{node {b}}
            child{node {c}};
        \end{tikzpicture}
        \hspace{0.2cm}
        \begin{tikzpicture}[sibling distance=5em, every node/.style = {shape=rectangle, rounded corners, draw, align=center,
                            top color=white, bottom color=blue!20}], left]

          \only<5> {
            \node{d}
            child{node{e}
              child{node{\color{red}f}}
            }
            child{node{g}};
          }

          \only<6> {
            \node{\color{red}d}
              child{node{\color{red}e}
                child{node{\color{red}f}}
              }
            child{node{g}};
          }
          \only<7> {
            \node{g};
          }
        \end{tikzpicture}
      \end{figure}
            
    }
  \end{frame}

  % Slide 9: Explication du calcul pour l'existence du coup gagnant dans la méthode Ulehla
  \begin{frame}
    \frametitle{La méthode Ulehla : l'existence du coup gagnant} 
    \only<1> {
      Commençons par prendre une forêt quelconque et regardons chaque arbre comme un graphe.
    }

    \only<2-5> {
      Colorons la forêt et supprimons les nœuds appartenant au noyau
      jusqu’à obtenir une forêt vide et attribuons une valeur de Rip pour chaque forêt obtenue.
    }
    \only<1-2> {
      \begin{figure}[h]
        \centering
        \begin{tikzpicture}[sibling distance=5em, every node/.style = {shape=rectangle, rounded corners, draw, align=center,
          top color=white, bottom color=blue!20}, right]

          \only<1> {
            \node{a}
              child{node{b}}
              child{node{c}};
          }

          \only<2> {
            \node {\color{black}a}
              child{node {\color{white}b}}
              child{node {\color{white}c}};
          }

        \end{tikzpicture}
        \hspace{0.2cm}
        \begin{tikzpicture}[sibling distance=5em, every node/.style = {shape=rectangle, rounded corners, draw, align=center,
          top color=white, bottom color=blue!20}], left]
          \only<1> {
            \node{d}
              child{node{e}
                child{node{f}}
              }
              child{node{g}};
          }
          \only<2> {
            \node{d}
              child{node{e}
                child{node{\color{white}f}}
              }
              child{node{\color{white}g}};
          }
        \end{tikzpicture}
        \only<2> {\caption {F, Rip(F) = 0}}
      \end{figure}
    }

    \only<3> {
      \begin{figure}
        \centering
        \begin{tikzpicture}[sibling distance=5em, every node/.style = {shape=rectangle, rounded corners, draw, align=center,
          top color=white, bottom color=blue!20}], left]

          \node{\color{white}a};
        
        \end{tikzpicture}
        \hspace{0.2cm}
        \begin{tikzpicture}[sibling distance=5em, every node/.style = {shape=rectangle, rounded corners, draw, align=center,
          top color=white, bottom color=blue!20}], left]
        
          \node{d}
            child{node{\color{white}e}};
        
        \end{tikzpicture}
        \caption{L(F), Rip(L(F)) = 1}
      \end{figure}
    }

    \only<4> {
      \begin{figure}
        \begin{tikzpicture}[sibling distance=5em, every node/.style = {shape=rectangle, rounded corners, draw, align=center,
          top color=white, bottom color=blue!20}], left]
        
          \node{\color{white}d};
        
        \end{tikzpicture}
        \caption{$l^2(F)\ rip(l^2(F)) = 1$}
      \end{figure}
    }

    \only<5> {
      \begin{figure}
        \begin{tikzpicture}[sibling distance=5em, every node/.style = {shape=rectangle, rounded corners, draw, align=center,
          top color=white, bottom color=blue!20}], left]
        
          \node{};
        
        \end{tikzpicture}
        \caption{$l^3(F)\ rip(l^2(F)) = 0$}
      \end{figure}
    }

    \only<5> {
      Pour une forêt donnée, la valeur de son nimber est égale à la suite des Rip de ses dénoyautages consécutifs (la valeur de Nimber de cette forêt est égale à 0110 = 6 en binaire).
    }

    \only<6> {
      \begin{block}{L'essentiel}  
        \begin{itemize}
          \item adaptation du théorème de Sprague-Grundy
          \item la valeur du nimber d'une forêt est son Rip
          \item l'addition de Nim est la suite des Rip d'une forêt 
        \end{itemize}
      \end{block}

      Mais comment savoir quel est ce coup gagnant ?
    }
  \end{frame}

  % Slide 10 : Explication du calcul du coup gagnant dans la méthode Ulehla
  \begin{frame}
    \frametitle{Methode Ulehla : le calcul du coup gagnant}
    \only<1> {
      \begin{block}{}  
        En plus d'adapter le théorème de Sprague-Grundy, la méthode de Ulehla permet de calculer le coup gagnant quand il existe.
      \end{block}
    }

    \only<2> {
      \begin{block}{}  
        Comme la dernière forêt ayant une valeur de Rip égale à 1 est la dernière forêt qui n'est pas un coup gagnant, il faut choisir une racine blanche ou l'un de ses successeurs directs blanc et la supprimer au \textit{sens du jeu} afin d'avoir une valeur de Rip égale à 0.
      \end{block}
    }

    \only<3-4> {
      \begin{block}{}  
        On remonte la liste des forêts et on réitère le procédé. 
      \end{block}
    }

    \only<2> {
      Dans l'exemple précédent, on doit chercher dans la forêt $l^2(F)$ et on choisi le nœud \textit{d} qui nous permet d'avoir un Rip égal à 0.
      \begin{figure}
        \begin{tikzpicture}[sibling distance=5em, every node/.style = {shape=rectangle, rounded corners, draw, align=center,
          top color=white, bottom color=blue!20}], left]
        
          \node{\only<2>{\color{white}}d};
        
        \end{tikzpicture}
        \caption{$l^2(F)\ rip(l^2(F)) = 1$}
      \end{figure}
    }

    \only<3> {
      On choisi entre le nœud choisi précédemment et un de ses successeurs directs blanc dans la forêt actuelle; ici si on supprime le nœud \textit{d}, alors on obtiendra bien une valeur de Rip égale à 0.

      \begin{figure}
        \centering
        \begin{tikzpicture}[sibling distance=5em, every node/.style = {shape=rectangle, rounded corners, draw, align=center,
          top color=white, bottom color=blue!20}], left]

          \node{\color{white}a};
        
        \end{tikzpicture}
        \hspace{0.2cm}
        \begin{tikzpicture}[sibling distance=5em, every node/.style = {shape=rectangle, rounded corners, draw, align=center,
          top color=white, bottom color=blue!20}], left]
        
          \node{d}
            child{node{\color{white}e}};
        
        \end{tikzpicture}
        \caption{L(F), Rip(L(F)) = 1}
      \end{figure}
    }

    \only<4> {
    Arrivé à la forêt initiale, on peut bien voir que si on supprime le nœud \textit{g}, alors on aura une valeur de Rip égale à 0.

    \begin{figure}[h]
        \centering
        \begin{tikzpicture}[sibling distance=5em, every node/.style = {shape=rectangle, rounded corners, draw, align=center,
          top color=white, bottom color=blue!20}, right]

          \node {\color{black}a}
            child{node {\color{white}b}}
            child{node {\color{white}c}};
          
        \end{tikzpicture}
        \hspace{0.2cm}
        \begin{tikzpicture}[sibling distance=5em, every node/.style = {shape=rectangle, rounded corners, draw, align=center,
          top color=white, bottom color=blue!20}], left]
  
          \node{d}
            child{node{e}
              child{node{\color{white}f}}
            }
            child{node{\color{white}g}};
        
        \end{tikzpicture}
       \caption {F, Rip(F) = 0}
      \end{figure}

      C'est pourquoi dans cette forêt, le coup gagnant est de supprimer le noeud \textit{g} de la forêt.
    }
  \end{frame}

  \section{Apports du stage}
  \label{sec:apports_du_stage}
  
  % Slide 11 : Explication des apports techniques du stage
  \begin{frame}
    \frametitle{Les apports techniques}
    
    \begin{block}{Nouveaux langages et technique de programmation}  
      \begin{itemize}
        \item le python
        \item une nouvelle approche de la programmation orientée objet
        \item une nouvelle approche de la programmation fonctionnelle
      \end{itemize}
    \end{block}
    \pause
    \begin{block}{Méthode de travail}
      \begin{itemize}
        \item partage hebdomadaire du travail
        \item recherches approfondies et mise en place d'un plan de travail
        \item appréhension d'un travail à découvrir (jeu de Chomp revisité)
        \item méthode de recherche et mise en forme d'une conjecture
      \end{itemize}
    \end{block}
  \end{frame}

  % Slide 12 : Explication des connaissances acquises sur le métier d'enseigant-chercheur
  \begin{frame}
    \frametitle{Les apports sur le métier d'enseignant-chercheur}
    \begin{block}{Les avantages}
      \begin{itemize}
        \item un métier en constante évolution
        \item un apport en connaissances constant
        \item la possibilité d'enseigner à un publique passionné
        \item une flexibilité sur les horaires
      \end{itemize}
    \end{block}
    \pause
    \begin{block}{Les à côtés du stage}
      \begin{itemize}
        \item la conférence de NORMASTIC
        \item une conférence sur les posset et la preuve de la conjecture de Dumont
        \item la répétition de la soutenance de thèse de M. Ali Chouria
        \item la soutenance de thèse de M. Ali Chouria
      \end{itemize}
    \end{block}
  \end{frame}

  % Slide 13 : Explications des choix d'orientation choisi grâce au stage
  \begin{frame}
    \frametitle{Les choix pour l'orientation}

    \begin{block}{Filière}  
      Grâce à ce stage et mon projet annuel portant sur la sécurité informatique, mon choix se porte sur la filière SSI. Cependant, l’idée de faire le double cursus SSI/ITA pour le premier semestre est très intéressante. 

      La thèse en milieu professionnel (thèse Cifre) est toujours envisage si la possibilité s'offre à moi.
    \end{block}
    \pause
    \begin{block}{Le monde professionnel}  
      Malgré énormément d'avantages dans le milieu universitaire, il reste encore trop de désavantages sur le nombre de postes, \textit{l'obligation} de quitter la région et le manque de financement dû à la politique actuelle des universités.
    \end{block}
  \end{frame}

  % Slide 14 : Fin de la présentation
  \begin{frame}
    \begin{figure}
      \centering
      Merci de votre attention
    \end{figure}
  \end{frame}
  
\end{document}
