\documentclass{beamer}

\usepackage[utf8]{inputenc}
\usepackage[T1]{fontenc}
\usepackage[francais]{babel}
\usepackage{tikz}
\usepackage{graphicx}

\usetikzlibrary{arrows,automata}
\usetikzlibrary{mindmap}

\graphicspath{{img/}}

\usetheme{Singapore}
% \usetheme{Warsaw}

\title{Stage de Licence 3 au sein du département informatique de l’université de Rouen}

\author[Tristan Rodriguez]{Tristan Rodriguez}
\institute{Université de Rouen, L3 Info}
\date{31/08/2016}

\setbeamersize{text margin left=0.5cm, text margin right=0.5cm}
\setbeamertemplate{blocks}[rounded][shadow=true]
\setbeamertemplate{itemize item}[triangle]
\beamertemplatetransparentcovered
\addtobeamertemplate{footline}{\insertframenumber/\inserttotalframenumber}

\begin{document}
  \section{Introduction}
  \label{sec:Introduction}


  % Slide 1 : Titre (présentation des conditions d'acquisition du stage et de Madame Selmi et Monsieur Dubernard)
  \begin{frame}
    \titlepage
  \end{frame}

  % Slide 2 : Table des matières (présentation rapide du travail effectué)
  \begin{frame}
    \frametitle{Plan de la présentation}
    \tableofcontents
  \end{frame}

  %Slide 3 : Présentation du département info
  \begin{frame}
    \frametitle{présentation du département informatique}
    
    \begin{center}
      \includegraphics[width=0.3\textwidth, height=0.1\textwidth]{logo_univ.png}  
    \end{center}

    \begin{block} {effectif du département}
      \begin{itemize}
        \item une trentaine d'enseignant chercheurs
        \item une vingtaine d'intervenant professionnels
        \item dix salles d'ordinateurs en réseau géré par Monsieur Macadré
      \end{itemize}
    \end{block}
    
    \pause

    \begin{block} {les thèmes de recherche du département}
      \begin{itemize}
        \item la combinatoire algebrique, enumerative et des mots
        \item la theorie des langages
        \item les automates fini
        \item le calcul symbolique
      \end{itemize}
    \end{block}
  \end{frame}

  % Slide 4 : présentation des formations du département
  \begin{frame}
    \frametitle{présentation des formations du département informatique}

    \begin{block} {les formations principales}
      \begin{itemize}
        \item une licence informatique
        \item un master spécialisé sur le développement logiciel (GIL)
        \item un master spécialisé sur la sécurité informatique (SSI)
        \item un master spécialisé sur la recherche informatique (ITA)
        \item des encadrement pour le doctorat (3 bourses obtenues pour l’année 2016-2017)
      \end{itemize}
    \end{block}
  \end{frame}
  % Slide 5 : Présentation des caractéristiques des jeux de Nim et de plusieurs jeux de nims connus
  \begin{frame}
    \frametitle{Un jeu de Nim, qu'est-ce que c'est ?}

    \begin{block} {c'est un jeu qui :}
      \begin{itemize}
        \item se joue a deux joueur 
        \item ne fait pas intervenir le hasard
        \item ne peux pas terminer sur une égalité
      \end{itemize}
    \end{block}
    
    \pause

    \begin{block} {jeux de Nim connus}
      \begin{itemize}
        \item jeu du tas d’allumettes résolu par Charles Bouton en 1901
        \item jeu des tas d'allumettes résolu indépendamment par Roland Sprague en 1935 et Patrick Grundy en 1939
      \end{itemize}
    \end{block}
  \end{frame}

  \section{travail théorique}
  \label{sec: travail théorique}

  %Slide 6 : presentation de la methode de resolution des jeux de Nim simple
  \begin{frame}
    \frametitle{résolution d'un jeu d'un tas de 3 d'allumettes (Bouton)}
    \only<1> {commençons par dessiner le graphe du jeux}
    \only<2> {donnons un \textit{nimber} a chaque nœuds}
    \only<2> {\alert{le nimber d'une position n'est autre que le nombre d'allumettes a cette position}}
    \only<3> {comme le nimber de la situation de départ n'est pas nul, il existe une stratégie gagnante a partir de ce nœud}
    \only<4> {nous cherchons maintenant comment obtenir un nimber nul a partir de cette position}
    \only<5> {Dans cette situation simple, pour être sur de gagner il faut donc de retirer toutes les allumettes des le début du jeu.}

    \only<1-4> {
      \begin{figure}[h]
        \centering
        \begin{tikzpicture}[->,>=stealth',shorten >=1pt,auto,node distance=3.5cm, semithick]
          \tikzset{
            state/.style = {shape=rectangle, rounded corners, draw, align=center, top color=white, bottom color=blue!20, text centered}
          }

          \node[initial, state] (3)                    {$3\ allumettes\ \only<2-3>{\color{red}3}$};
          \node[state]          (f) [below of=3]       {$fin\ du\ jeu \ \only<2>{\color{red}0}\only<4>{\color{red}0}$};
          \node[state]          (2) [below right of=f] {$2\ allumettes\ \only<2>{\color{red}2}$};
          \node[state]          (1) [below left of=f]  {$1\ allumette\ \only<2>{\color{red}1}$};
          
          \path (3) edge (2)
                (3) edge (1)
                (3) edge (f)
                (2) edge (1)
                (2) edge (f)
                (1) edge (f);    
          \only<4->{\color{red} \path (3) edge (f);}  
        \end{tikzpicture}
      \end{figure}
    }
  \end{frame}

  % Slide 7 : presentation du theoreme de Sprague-Grundy
  \begin{frame}
  \frametitle{résolution d'un jeu de Nim par le théorème de Sprague-Grundy}
    \only<1> {La méthode repose sur le travail de Bouton : prenons un jeu avec deux tas donc chacun a 2 allumettes}

    \only<2> {comme on ne peux retirer que dans un tas a la fois, le graphe devient plus grand}
    \only<3> {on attribue a nouveau les nimbers comme la solution de Bouton}
    \only<4> {comme le nimber de la situation de départ est nul, il n'existe pas de stratégie gagnante pour ce joueur}
    \only<5> {cette solution nous montre que si l'on joue face a quelqu'un qui connait la méthode pour gagner, nous perdrons a tous les coups si nous jouons en premier}
    \only<2-4> {
      \begin{figure}[h]
        \centering
        \begin{tikzpicture}[->,>=stealth',shorten >=1pt,auto,node distance=2.5cm, semithick]
          \tikzset{
            state/.style = {shape=rectangle, rounded corners, draw, align=center, top color=white, bottom color=blue!20, text centered}
          }

          \node[initial, state] (22)                     {$2,2\ \only<3-4>{\color{red}0}$};   
          \node[state]          (21) [below left of=22]  {$2,1\ \only<3>{\color{red}3}$};
          \node[state]          (11) [below left of=21]  {$1,1\ \only<3>{\color{red}0}$};
          \node[state]          (2)  [below right of=22] {$2\ \only<3>{\color{red}2}$};
          \node[state]          (1)  [below of=21]       {$1\ \only<3>{\color{red}1}$};
          \node[state]          (f)  [below of=2]        {$final\ \only<3>{\color{red}0}$};
          
          \path (22) edge (21)
                (22) edge (2)
                (21) edge (2)
                (21) edge (1)
                (21) edge (11)
                (11) edge (1)
                (2) edge (1)
                (2) edge (f)
                (1) edge (f);        
        \end{tikzpicture}
      \end{figure}
    }


    \only<3> {le nimber d'une position est égal au \textit{ou exclusif} du nombre d'allumettes de chaque tas a cette position}
  \end{frame}

  % Slide 8 : présentation du jeu de Hackendot et bref historique de travail dessus
  \begin{frame}
    \frametitle{Le Hackendot qu'es-ce que c'est?}
    \only<1> {
      \begin{block}{quelques informations}  
        \begin{itemize}
          \item un jeu qui se joue sur une foret (un ensemble d'arbre)
          \item la solution de ce jeu a été proposé par J. Ulehla en 1979
          \item la solution est une application du théorème de Sprague-Grundy a ce jeu adapté pour ce jeu
        \end{itemize}
      \end{block}
    }

    \only<2-> {
      \begin{block}{les règles du jeu}  
        \begin{itemize}
          \pause
          \item on choisi un arbre présent dans la foret du jeu
          \pause
          \item on choisi un nœud dans l'arbre précédemment choisi
          \pause
          \item on retire tout le chemin menant du nœud choisi a la racine de l'arbre choisi 
        \end{itemize} 
      \end{block}
    }

    \only<5-> {
      exemple :
      \begin{figure}[h]
        \centering
        \begin{tikzpicture}[sibling distance=5em, every node/.style = {shape=rectangle, rounded corners, draw, align=center,
                      top color=white, bottom color=blue!20}, right]

          \node {a}
            child{node {b}}
            child{node {c}};
        \end{tikzpicture}
        \hspace{0.2cm}
        \begin{tikzpicture}[sibling distance=5em, every node/.style = {shape=rectangle, rounded corners, draw, align=center,
                            top color=white, bottom color=blue!20}], left]

          \only<5> {
            \node{d}
            child{node{e}
              child{node{\color{red}f}}
            }
            child{node{g}};
          }

          \only<6> {
            \node{\color{red}d}
              child{node{\color{red}e}
                child{node{\color{red}f}}
              }
            child{node{g}};
          }
          \only<7> {
            \node{g};
          }
        \end{tikzpicture}
      \end{figure}
            
    }
  \end{frame}

  \section{Apports du stage}
  \label{sec:apports_du_stage}
  
  \begin{frame}
    \frametitle{La méthode Ulehla : l'existence du coup gagnant} 
    \only<1> {
      commençons par prendre une foret quelconque et regardons chaque arbres comme un graphe
    }

    \only<2-5> {
      Colorons la foret et supprimons les nœuds appartenant au noyau
      jusqu’à obtenir une foret vide et attribuons une valeur de Rip pour chaque foret obtenu
    }
    \only<1-2> {
      \begin{figure}[h]
        \centering
        \begin{tikzpicture}[sibling distance=5em, every node/.style = {shape=rectangle, rounded corners, draw, align=center,
          top color=white, bottom color=blue!20}, right]

          \only<1> {
            \node{a}
              child{node{b}}
              child{node{c}};
          }

          \only<2> {
            \node {\color{black}a}
              child{node {\color{white}b}}
              child{node {\color{white}c}};
          }

        \end{tikzpicture}
        \hspace{0.2cm}
        \begin{tikzpicture}[sibling distance=5em, every node/.style = {shape=rectangle, rounded corners, draw, align=center,
          top color=white, bottom color=blue!20}], left]
          \only<1> {
            \node{d}
              child{node{e}
                child{node{f}}
              }
              child{node{g}};
          }
          \only<2> {
            \node{d}
              child{node{e}
                child{node{\color{white}f}}
              }
              child{node{\color{white}g}};
          }
        \end{tikzpicture}
        \only<2> {\caption {F, Rip(F) = 0}}
      \end{figure}
    }

    \only<3> {
      \begin{figure}
        \centering
        \begin{tikzpicture}[sibling distance=5em, every node/.style = {shape=rectangle, rounded corners, draw, align=center,
          top color=white, bottom color=blue!20}], left]

          \node{\color{white}a};
        
        \end{tikzpicture}
        \hspace{0.2cm}
        \begin{tikzpicture}[sibling distance=5em, every node/.style = {shape=rectangle, rounded corners, draw, align=center,
          top color=white, bottom color=blue!20}], left]
        
          \node{d}
            child{node{\color{white}e}};
        
        \end{tikzpicture}
        \caption{L(F), Rip(L(F)) = 1}
      \end{figure}
    }

    \only<4> {
      \begin{figure}
        \begin{tikzpicture}[sibling distance=5em, every node/.style = {shape=rectangle, rounded corners, draw, align=center,
          top color=white, bottom color=blue!20}], left]
        
          \node{\color{white}d};
        
        \end{tikzpicture}
        \caption{$l^2(F)\ rip(l^2(F)) = 1$}
      \end{figure}
    }

    \only<5> {
      \begin{figure}
        \begin{tikzpicture}[sibling distance=5em, every node/.style = {shape=rectangle, rounded corners, draw, align=center,
          top color=white, bottom color=blue!20}], left]
        
          \node{};
        
        \end{tikzpicture}
        \caption{$l^3(F)\ rip(l^2(F)) = 0$}
      \end{figure}
    }

    \only<5> {
      La valeur du Nimber de cette foret est la suite des rip de ces forets, donc on a un coup gagnant dans cette foret (valeur de Nimber de cette foret égal a 0110 = 6 en binaire).
    }

    \only<6> {
      \begin{block}{l'essentiel}  
        \begin{itemize}
          \item adaptation du théorème de Sprague-Grundy
          \item la valeur du nimber d'une foret est son rip
          \item l'addition de Nim est la suite des rip d'une foret 
        \end{itemize}
      \end{block}

      Mais comment savoir quel est ce coup gagnant ?
    }
  \end{frame}

  \begin{frame}
    \frametitle{Methode Ulehla : le calcul du coup gagnant}
    \only<1> {
      \begin{block}{}  
        En plus d'adapter le theoreme de Sprague-Grundy, la methode de Ulehla permet de calculer le coup gagnant quand il existe
      \end{block}
    }
  \end{frame}
\end{document}